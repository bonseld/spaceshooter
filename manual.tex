\documentclass{article}

\usepackage{graphicx}

\usepackage{fancyvrb}
\usepackage{amsmath}


\title{Space Shooter}
\author{Dylan Bonsell}
\date{Spring 2014}


\begin{document}
\maketitle
CS321
Matthews
\newpage
\section{Back Story}
	Your are a ship on a mission, to destroy all other ships!

\newpage
\section{User's Guide}
	To move, move the mouse! To shoot, press the space bar. To continue to shoot, hold the space bar.
	Collect power up stars to upgrade your gun!
	If ships get past you, you lose 10 points. If you kill them, you gain 5.
\newpage
\section{Module Documentation}
	spritesheet is a basic sprite sheet loader, imports files, and gets images at locations. It also has the ability to load a strip.
\\
	There are classes for Bullet, EnemyBullet, Ship, Powerup, and Enemy.
	Each of these have an update, which handles itself in moving.
\\ 
	There is a Menu and MenuItem Class, which was taken from a basic menu tutorial. They essentially give the ability to load a menu on the fly, and handle the objects easily.
\newpage
\section{Cheats}
	There are no cheats, as there is no end game! The game simply gets harder the longer you play
\newpage
\section{Acknowledgement}
\begin{verbatim}
Menu:
http://scripters-corner.net/2013/04/11/creating-a-menu-in-pygame/
Slow down sound:
http://www.freesound.org/people/RICHERlandTV/sounds/216093/
SpriteSheet
http://www.pygame.org/wiki/Spritesheet?parent=CookBook
Music:
https://www.youtube.com/watch?v=S0ofoYDPNMY
Explosion:
http://spritedatabase.net/files/snes/344/Sprite/Spaceship.png
Star for Power Up:
http://awcomics.wikia.com/wiki/Starter_Sprites
Background:
http://walls4joy.com/wallpaper/296556-outer-space-stars
Random sounds are generated with this:
http://www.bfxr.net/
Enemy Ships:
http://opengameart.org/sites/default/files/ships.png
Main Ship:
http://www.blitzbasic.com/Community/posts.php?topic=86289
\end{verbatim}
\newpage
\section{Autobiographical}
I have a lot of python experience, but visual rendering is new to me. The most fun was getting just simple sprites to work, and then actually move the way I wanted them too. The rules were simple, but using an OOP design is odd but useful.
One of the issues I has was when setting $key.set_repeat$ it removes the key being held down. I have yet to find a way around other that to let up on the space bar and put it back down again.
\end{document}
